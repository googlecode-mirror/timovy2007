% $Id: demo.tex,v 1.4 2007-02-21 15:30:08 kosik Exp $

\documentclass{iitsrc}[2006/14/02]
\usepackage[utf8x]{inputenc}
\usepackage{graphicx}

% Begin --ADDED PACKAGES-- %
\usepackage{listings} % source code listing 
\usepackage{url} 
% End --ADDED PACKAGES-- %

% Please do not remove the following command:
\editpages{1}{8}

\title{Device Drivers in Minix3}
\author{Peter}{C\'ich}
\author{Pavol}{F\'abik}
\author{J\'an}{Garaj}
\author{Jozef}{Hergott}
\author{Jozef}{Hopko}
\supervision{doc. Ing. Jana Min\'arov\'a, PhD.}
            {\fiit.}
% Within the `affiliation' parameter you can use the `\fiit' macro to
% generate the full name of our faculty.
\mail{jozef.hopko@gmail.com}

%
\newcommand\fig[4]{%
	\begin{figure}
	\begin{center}
	\includegraphics[scale=#1]{#2}
	~\\%
	\caption{#4}
	\label{#3}
	\end{center}
	\end{figure}

\begin{document}

\begin{abstract}
Information system for communication with graduates represents only one of many ways how can university stay in touch with its graduates. In addition to communication between university and graduate the information system allows communication between graduates, their personal presentation for public. System also collects actual information about working experiences of graduates, which can be used to improve faculty credits and teaching processes. Presented information system includes all this points with focus on security, usability and comfortable user interface.
\end{abstract}

%%%%%%%%%%%%%%%%%%%%%%%%%%%%%%%%%%%%%%%%%%%

\section{Introducing}

Communication between graduates and university can be an advantage for boht sides. In this article we would shortly introduce web based software system called ALUMNI developed on Team project. 

Our faculty has ambitions to present graduates to public society. The faculty also wants to stay in touch with graduates using web application, providing a channel for profesional and social communication between graduates each other and faculty. 

The focus of ALUMNI project is to design and implement a system that would accomplish this needs. 

Graduate FIIT STU is person, which has been studying for minimal three years or eventually more at this faculty. During this period the graduaed tied up a lot of social contact with classmates as well as with teachers. These contacts are typically cut off after graduating or the only rarely continue.

Generally speaking, after the student leaves the university, he lost most of the contacts. This trend of breaking all contacts is not wanted by the faculty. The faculty sees the graduates as potential bussines partners with possibility to cooperated after study. Graduates are very good source of information and critics that could help inproove the quality of faculty. The best way to gain required information are feedback forms.

%* dalsia motivacia, vyhody pre fakultu

%Absolvent FIIT STU je osoba, ktorá minimálne tri roky, prípadne viac, navštevovala túto fakultu. Počas tohto obdobia nadviazal absolvent množstvo sociálnych kontaktov so svojimi spolužiakmi a tiež so vyučujúcimi. Po ukončení sa tieto kontakty zvyčajne pretrhnú a len málokedy pokračujú ďalej. Predovšetkým tie s vyučujúcimi, všeobecne povedané absolvent stratí akékoľvek kontakty s fakultou. Táto skutočnosť však nie je v záujme fakulty a s veľkou pravdepodobnosťou k tomuto zisteniu dospeje aj absolvent vo svojom ďalšom živote. Tento trend, pretrhnutie kontaktov, je považovaný na fakulte za nežiadúci. Fakulta totiž vidí vo svojich absolventoch ľudí, s ktorými chce spolupracovať aj po ukončení štúdia, chce od nich získavať podnety a pripomienky, ktoré by premietla do svojho budúceho smerovania a do svojho zlepšovania. Ako najprijateľnejšiu formu kooperácie absolventov so fakultou sa z pohľadu fakulty javí spätná väzba a získavanie priamych informácii a názorov od absolventov. K naplneniu tejto vízie zo strany fakulty by mal prispieť aj navrhovaný systém na prezentáciu absolventov ALUMNI,

\section{Bussines goals}
Ciele systému:
Prezentovať základné informácie o absolventoch verejnosti.
Znamená to zabezpečiť vytvorenie a udržiavanie databázy absolventov a vhodne prezentovať základné informácie o jednotlivcovi verejnosti na webe bez obmedzenia prístupu. Predpokladáme stručnú informáciu o absolventovi, kedy študoval, absolvoval, jeho špecializácia, o téme jeho bakalárskej, prípadne diplomovej práce, prípadne abstrakt práce. Tu by bolo vítané rozšíriť prezentáciu o grafické vyjadrenie zamestnanosti a odbornosti absolventov z rôznych hľadísk, pokiaľ dokážeme od nich získať k tomu potrebné údaje.
Sprostredkovať fakulte získavanie aktuálnych informácií o absolventoch v praxi.
Ide o aktualizáciu kontaktu, zamestnania, profesijného vývoja, odborného zamerania, oblasti, v ktorej je aktívny, sfér odborného záujmu a pod., tie, ktoré poskytne sám absolvent. Táto oblasť je veľmi citlivá, vyžaduje záujem o kontakt z oboch strán a je podmienená prísnou ochranou údajov s vhodne zorganizovaným autorizovaným prístupom. Je to dôležitá, pre fakultu užitočná úloha, treba ju uvažovať.
Umožniť absolventom vzájomnú komunikáciu.
Má to byť jednoduchá a bezpečná komunikácia v informatickej komunite chránená starostlivo navrhnutými prístupovými právami pre skupiny autorizovaných účastníkov. Má slúžiť na neformálnu výmenu informácií v komunite kolegov, rovesníkov, odborníkov z praxe, ktoré komunite poskytne sám účastník. Okrem sprostredkovania kontaktu môže byť úlohou tejto časti systému informovať záujemcov o odborných aktivitách komunity, poskytnúť pre ne priestor – fórum, prípadne ďalšie vhodné činnosti.

obrazok. vztah fakulta alumnus, naznacenie komunikacie

- graduates view (graduates expectations and requirements)
- faculty view (faculty expectations and requirements)

\section{System ALUMNI}

- najdolezitesie poskytovane funkcie, zaujime problemy ktore sme museli vyriesit, co sme zmenili, co pripravilo problemy atd.
Autentifikacia
Komunikacia
Prezentacia

\section{Architecture}
%  TODO trosku popisat architekturu systemu, na com je zalozeny, MVC-CakePHP, databaza
Cely navrh architektury vychadza z poziadavky co najvacsej dostupnosti pre verejnost a aj absolventov. Tuto poziadavku relativne dobre splna architektura klient-server postavena na webovych technologiach. Cela interakcia so systemom prebieha vo webovom prehliadaci, ktory je beznou sucastou operacnych systemov a nijak nezavisi na aplikacnej casti na serveri. Cela biznis logika a funcionalita je implementovana priamo na serveri, pricom je zavisla od konkretnej implementacie. 
Logicky je system na strane servera rozdeleny do viacerych modulov, ktore spolocne poskytuju kompletnu funkcnost systemu. Vsetky moduly su vytvorene nas PHP frameworkom CakePHP, ktory je navrhnuty na rapid develepment. Tento  framework nam ulahcuje pracu s celym systemom, nakolko riesi standardne problemy vo vyvoji webovych aplikacii ako napr. autentifikaciu, pristup k databaze,  prezentaciu a pod.
CakePHP plne podporuje MVC (Model-View-Controller) model. Tento model 
predstavuje rozdelenie systemu do troch funkcnych casti. Model, View a Controller. Vztahy medzi jednotlivymi castami su zobrazene na obrazku c. 1.
% TODO cislovanie obrazka
% TODO vlozit obrazok MVC-CakePHP
% TODO ako zvyraznit subnadpis?
Model
Prestavuje cast, ktora sa stara o ziskavanie a ukladanie informacii. Mysli sa tym samotna databaza, pripadne obsluzne skripty, ktore vykonavaju akcie nad touto databazou. Udaje ziskane z casti Model su nasledne poskytovane casti View.
View
Tato cast systemu sa stara o prezentaciu dat pouzivatelovi. Ide o pouzivatelske rozhranie (user interface). V pripade webovych systemov predstavuje View konkretny HTML vystup, ktory sa pouzivatelovi zobrazi v internetovom prehliadaci ako internetova stranka. Cast View predstavuje prezentacnu cast systemu, ktora moze obsahovat aj prezentacnu logiku.
Controller
Controller je cast, ktora sa stara o spravu akcii vykonanych pouzivatelom alebo systemom. Tieto akcie su nasledne prenesene ako zmeny v casti View alebo v casti Model. Pod zmenou v casti View sa mysli napriklad zmena stranky, pod zmenou v casti Model napriklad ulozenie novych informacii do databazy. Controller moze obsahovat aj takzvanu biznis logiku. 

% ukazkovy text ako citovat a odkazovat sa na literaturu
As mentioned before, Minix3 is microkernel based system. The authors intention was to keep the system "Small is Beautiful" \cite[page 17]{osdesign}.
It is composed from multiple processes. Each of these processes belongs to one of the four layers, see Figure~\ref{fig:minixstruct}. Processes in the first layer run in the privileged mode. They have unrestricted access to everything. This layer contains only three processes: {\em kernel task}, {\em system task} and the {\em clock task}.

% vlozenie obrazku
\fig{1}{images/fourlayers}{fig:minixstruct}{Internal structure of Minix3.}
%
\subsection{Podsekcia}

      
\section{Summary}

pokec o tom co sme tu popisali a preco je ALUMNI prinosny pre fakultu aj studenta 
  
%%%%%%%%%%%%%%%%%%%%%%%%%%%%%%%%%%%%%%%%%%%
\acknowledgement{This work was partially supported by the Institute of Informatics and Software Engineering, Faculty of Informatics and Information Technologies, Slovak University of Technology in Bratislava.}

\nocite{pcispec}
\nocite{ac97spec}
\nocite{ich4spec}
\nocite{crytalspec}
\nocite{osdesign}
\nocite{wdmdriver}
\nocite{alsadriver}
\nocite{pciwiki}

\bibliography{literature_article}
\bibliographystyle{iitsrc}

\end{document}

